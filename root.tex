\documentclass{article}

\author{Джоуни Суннари}
\title{Курсовик}
\usepackage[utf8]{inputenc}
\usepackage[russian]{babel}

\begin{document}
\maketitle{}

\tableofcontents{}

\section{Теоритический этап}
% определение понятий «информационная система»,
% используемых видов архитектур ИС, элементов, составляющих используемые виды
% архитектур, других специальных понятий, используемых в курсовой работе;

\subsection{Описание прикладных и бизнес-процессов}
%   В рамках тематики 1 предполагается описание деятельности предприятия в целом,
% его организационной структуры, выделение контуров и конкретизация задач управления
% для каждого контура (управление производством, управление кадрами, управление
% закупками и т.д.), а также указание основных информационных объектов, которые
% используются в каждом контуре (планы, графики, спецификации, договора, наряды,
% реестры и т.п.).
%   В рамках тематики 2 предполагается описание деятельности предприятия в целом,
% места в его деятельности выбранного контура управления, процессов, реализуемых в этом
% контуре и связанных с ними информационных процессов (какие процессы сбора,
% обработки, хранения, передачи и предоставления информации реализованы в этом
% контуре и насколько они формализованы), информационных объектов, которые
% используются в этих процессах (планы, графики, спецификации, договора, наряды,
% реестры и т.п.).
%   В рамках тематики 3 предполагается описание типовых информационных
% процессов (сбора, обработки, хранения, передачи и предоставления информации) для
% автоматизации которых предназначено определенное в теме средство автоматизации,
% конкретных информационных объектов, формируемых, хранимых, обрабатываемых или
% передаваемых этим средством, целей и показателей качества этих информационных
% процессов.

\subsection{Формализованное визуальное моделирование и формирование требований}
% Выбирается методология и в соответствии с ее правилами формируется набор
% диаграмм, дающих формальное описание процессов. Модели должны демонстрировать
% анализируемые процессы с точностью до отдельных операций, позволять для этих
% операций определить акторов и информационные объекты, использующиеся в них.
% Делаются выводы о функциональных требованиях к средствам автоматизации со стороны
% смоделированных процессов. Для второй тематики обязательно, а для первой и третьей
% тематики рекомендуется описать нефункциональные требования (к производительности,
% надежности, безопасности и т.п.) к средствам автоматизации и обосновать их исходя из
% приведенных выше описаний и моделей.

\section{Анализ и моделирование процессов}
% сбор, анализ и
% систематизация информации об объекте автоматизации в рамках выбранной тематики КР,
% моделирование деятельности предприятия, отдельного контура управления предприятием,
% процессов, автоматизируемых выбранным программным средством (средствами),
% формирование требований к средствам автоматизации

% Все процессы можно не брать
% Процессы - это как в первой лабе
% близко рассматривать можно только одну систему

\section{Анализ средств автоматизации процессов}
% сбор, анализ и
% систематизация информации о средствах автоматизации в рамках выбранной тематики
% КР: функциональных возможностях, реализуемых информационных объектах,
% требованиях к инфраструктуре и способах развертывания, программных компонентах и
% способах их взаимодействия, структурах данных и организации хранилищ данных и т.п.

%   В рамках тематики 1 предполагается структурированное описание типовых
% функциональных возможностей классов информационных систем, применяющихся для
% автоматизации определенных на предыдущем этапе процессов, обоснование выбора
% конкретного набора информационных систем, детальное описание их функциональных
% возможностей и сопоставление их с функциональными требованиями, полученными на
% предыдущем этапе.
%   В рамках тематики 2 предполагается структурированное описание
% функциональных возможностей одного или нескольких средств автоматизации,
% применяющихся для автоматизации определенных на предыдущем этапе процессов,
% обоснование возможности их применения (рекомендуется также обосновать выбор
% конкретных средств), сопоставление их функциональных возможностей с требованиями
% процессов, детальное описание требований к ИТ-инфраструктуре со стороны выбранных
% программных средств, рекомендуемых производителем вариантов развертывания этих
% средств и средств их интеграции между собой и с внешними системами.
%   В рамках тематики 3 предполагается структурированное описание программной
% архитектуры анализируемого программного средства на уровне выделения отдельных
% программных компонентов, библиотек, модулей, описания основных классов, логики их
% взаимодействия, а также описание архитектуры данных, включающее в себя описание
% используемых стандартных типов данных, сложных пользовательских типов данных,
% организации хранения структурированных данных в хранилищах. Для каждого
% программного компонента приводится описание его назначения, а также технологий,
% используемых при его разработке. Все описания обязательно дополняются визуальными
% моделями, построенными в соответствии с требованиями нотации UML.

\section{Синтез определенных уровней архитектуры ИС}
% (непосредственно проектирование архитектуры ИС на уровне (уровнях), определяемых
% тематикой КР: функциональной архитектуры, информационной архитектуры, системнойархитектуры, программной архитектуры, архитектуры данных, обоснование соответствия
% построенной архитектуры требованиям процессов).


% В рамках тематики 1 предполагается представление функциональной и
% информационной архитектуры ИС предприятия, включающей все выбранные на
% предыдущем этапе программные средства автоматизации. Функциональная архитектура
% представляется как распределение операций смоделированных процессов по
% функциональным компонентам отдельных программных средств. В случае
% взаимосвязанных процессов или распределения операций одного процесса по нескольким
% средствам автоматизации указывается передача данных между функциональными
% компонентами соответствующих систем. Информационная архитектура представляется в
% виде сопоставления информационных объектов, выделенных на первом этапе синформационными объектами, реализованными в выбранных средствах автоматизации.
% Также требуется описание интеграции систем на уровне совместного использования
% преобразования данных информационных объектов, обеспечение целостности данных и
% синхронизации выполняемых над ними операций.
%   В рамках тематики 2 предполагается построение системной архитектуры.
% Необходимо обосновать выбор способа развертывания определенного на предыдущем
% этапе средства автоматизации (или нескольких), включая обоснование количество и
% размещение серверных компонентов, в том числе с использованием технологий
% виртуализации, выбор телекоммуникационных технологий, операционных систем,
% базового программного обеспечения. В описание полученной системной архитектуры
% включаются аппаратные узлы для размещения как серверных, так и клиентских
% компонентов системы, периферийное оборудование, телекоммуникационное
% оборудование, любое системное или вспомогательное программное обеспечение.
% Обосновывается соответствие построенной архитектуры функциональным и
% нефункциональным требованиям, определенным на первом этапе и требованиям к ИТ-
% инфраструктуре со стороны программных средств автоматизации, указанным на втором
% этапе. Указывается за счет каких архитектурных решений обеспечивается требуемый
% уровень производительности (в том числе в условиях изменяющейся нагрузки),
% надежности и безопасности ИС. Описание спроектированной архитектуры
% сопровождается диаграммами, выполненными в соответствии с требованиями нотации
% UML.
%    рамках тематики 3 предполагается сопоставление функциональной,
% информационной, программной архитектуры и архитектуры данных для выбранного
% программного средства автоматизации информационных процессов. В результате должны
% быть построены как минимум два обязательных сопоставления. Первое – сопоставление
% функциональной и программной архитектуры. Должно быть указано, какие функции,
% относящиеся к каким функциональным компонентам, реализованы с помощью каких
% программных компонентов, а также какие программные компоненты используются
% одновременно для реализации различных функций. Второе – сопоставление
% информационной архитектуры и архитектуры данных. Должно быть указано, какие
% информационные объекты реализованы с помощью каких структур данных, как
% организовано хранение информационных объектов. Отдельно рекомендуется описать, как
% обеспечивается целостность данных при работе с информационными объектами,
% транзакционность выполнения операций, непротиворечивость и уникальность
% идентификаторов и другие технические аспекты реализации информационных объектов
% на уровне работы с данными.


\end{document}
